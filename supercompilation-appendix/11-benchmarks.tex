\section{Anti-KMP test}

This section lists in detail the results of the anti-KMP-test mentioned in the main text.
To improve readability, we set specifically for these examples \texttt{sizeBound=10}.

The baseline transformer produces 19 functions.
\begin{lstlisting}[style=demo,escapechar=!]
-- demo18
ghci> transform ({{even(sqr(x))}}, prog1)
f1(x)
f1(x) = g2(x, x);
f3() = True();
f6() = True();
f7(v2, v1, x) = g8(v2, v1, x);
f10() = False();
f12(v4, v3, x) = g13(v4, v3, x);
f15() = True();
f16(v3, v1, x) = g17(v3, v1, x);
f19() = True();
g2(Z(), x) = f3();
g2(S(v1), x) = g4(x, x, v1);
g4(Z(), x, v1) = g5(v1, x);
g4(S(v2), x, v1) = f7(v2, v1, x);
g5(Z(), x) = f6();
g5(S(v2), x) = g4(x, x, v2);
g8(Z(), v1, x) = g9(v1, x);
g8(S(v3), v1, x) = f16(v3, v1, x);
g9(Z(), x) = f10();
g9(S(v3), x) = g11(x, x, v3);
g11(Z(), x, v3) = g9(v3, x);
g11(S(v4), x, v3) = f12(v4, v3, x);
g13(Z(), v3, x) = g14(v3, x);
g13(S(v5), v3, x) = f7(v5, v3, x);
g14(Z(), x) = f15();
g14(S(v5), x) = g4(x, x, v5);
g17(Z(), v1, x) = g18(v1, x);
g17(S(v4), v1, x) = f7(v4, v1, x);
g18(Z(), x) = f19();
g18(S(v4), x) = g4(x, x, v4);
\end{lstlisting}

Deforestation -- 11 functions:
\begin{lstlisting}[style=demo,escapechar=!]
-- demo19
ghci> deforest ({{even(sqr(x))}}, prog1)
g1(x, x)
f4(v2, v1, x) = g5(v2, v1, x);
g1(Z(), x) = True();
g1(S(v1), x) = g2(x, x, v1);
g2(Z(), x, v1) = g3(v1, x);
g2(S(v2), x, v1) = f4(v2, v1, x);
g3(Z(), x) = True();
g3(S(v2), x) = g2(x, x, v2);
g5(Z(), v1, x) = g6(v1, x);
g5(S(v3), v1, x) = g10(v3, v1, x);
g6(Z(), x) = False();
g6(S(v3), x) = g7(x, x, v3);
g7(Z(), x, v3) = g6(v3, x);
g7(S(v4), x, v3) = g8(v4, v3, x);
g8(Z(), v3, x) = g9(v3, x);
g8(S(v5), v3, x) = f4(v5, v3, x);
g9(Z(), x) = True();
g9(S(v5), x) = g2(x, x, v5);
g10(Z(), v1, x) = g11(v1, x);
g10(S(v4), v1, x) = f4(v4, v1, x);
g11(Z(), x) = True();
g11(S(v4), x) = g2(x, x, v4);
\end{lstlisting}

%\newpage
Supercompilation -- 34 functions.
\begin{lstlisting}[style=demo,escapechar=!]
-- demo20
ghci> supercompile ({{even(sqr(x))}}, prog1)
g1(x)
g1(Z()) = True();
g1(S(v1)) = g2(v1);
g2(Z()) = False();
g2(S(v2)) = g3(v2);
g3(Z()) = True();
g3(S(v3)) = g33(S(g4(v3)));
g4(Z()) = S(S(S(S(S(S(Z()))))));
g4(S(v5)) = S(g32(v5, S(S(S(S(g31(v5, S(S(S(S(g30(v5, 
	S(S(S(S(g29(v5, g5(v5))))))))))))))))));
g5(Z()) = Z();
g5(S(v10)) = S(S(S(S(S(g28(v10, g6(v10)))))));
g6(Z()) = Z();
g6(S(v12)) = S(S(S(S(S(S(g27(v12, g7(v12))))))));
g7(Z()) = Z();
g7(S(v14)) = S(S(S(S(S(S(S(g26(v14, g8(v14)))))))));
g8(Z()) = Z();
g8(S(v16)) = g25(S(S(S(S(S(S(S(S(v16)))))))), 
	g9(v16, S(S(S(S(S(S(S(S(v16))))))))));
g9(Z(), v18) = Z();
g9(S(v19), v18) = g10(v18, v19);
g10(Z(), v19) = g11(v19);
g10(S(v20), v19) = S(g12(v20, v19));
g11(Z()) = Z();
g11(S(v20)) = g11(v20);
g12(Z(), v19) = g13(v19);
g12(S(v21), v19) = S(g14(v21, v19));
g13(Z()) = Z();
g13(S(v21)) = S(g13(v21));
g14(Z(), v19) = g15(v19);
g14(S(v22), v19) = S(g16(v22, v19));
g15(Z()) = Z();
g15(S(v22)) = S(S(g15(v22)));
g16(Z(), v19) = g17(v19);
g16(S(v23), v19) = S(g18(v23, v19));
g17(Z()) = Z();
g17(S(v23)) = S(S(S(g17(v23))));
g18(Z(), v19) = g19(v19);
g18(S(v24), v19) = S(g20(v24, v19));
g19(Z()) = Z();
g19(S(v24)) = S(S(S(S(g19(v24)))));
g20(Z(), v19) = g21(v19);
g20(S(v25), v19) = S(g24(v25, g22(v19, v25)));
g21(Z()) = Z();
g21(S(v25)) = S(S(S(S(S(g21(v25))))));
g22(Z(), v25) = Z();
g22(S(v27), v25) = S(S(S(S(S(S(g23(v25, g22(v27, 
	v25))))))));
g23(Z(), v28) = v28;
g23(S(v29), v28) = S(g23(v29, v28));
g24(Z(), v26) = v26;
g24(S(v27), v26) = S(g24(v27, v26));
g25(Z(), v17) = v17;
g25(S(v19), v17) = S(g25(v19, v17));
g26(Z(), v15) = v15;
g26(S(v16), v15) = S(g26(v16, v15));
g27(Z(), v13) = v13;
g27(S(v14), v13) = S(g27(v14, v13));
g28(Z(), v11) = v11;
g28(S(v12), v11) = S(g28(v12, v11));
g29(Z(), v9) = v9;
g29(S(v10), v9) = S(g29(v10, v9));
g30(Z(), v8) = v8;
g30(S(v9), v8) = S(g30(v9, v8));
g31(Z(), v7) = v7;
g31(S(v8), v7) = S(g31(v8, v7));
g32(Z(), v6) = v6;
g32(S(v7), v6) = S(g32(v7, v6));
g33(Z()) = True();
g33(S(v5)) = g34(v5);
g34(Z()) = False();
g34(S(v6)) = g33(v6);
\end{lstlisting}

We compare below the speed-ups of the task \texttt{{\color{brown}even(sqr(x))}, prog1}
obtained by using \texttt{transform}, \texttt{deforest}, and \texttt{supercompile}.
The speed-up of task $t_2$ with respect to task $t_1$ is calculated
as follows (where \texttt{x} is a natural number, and \texttt{{\color{brown}x}} -- the corresponding Peano number):

\begin{lstlisting}[style=demo]
acc t1 t2 x = steps1 / steps2 where
	(_, steps1) = sll_trace t1 [("x"), {{x}}]
	(_, steps2) = sll_trace t2 [("x"), {{x}}]
\end{lstlisting}

\begin{tikzpicture}[yscale=2,xscale=0.25]
%\draw[very thin,color=gray] (-0.1,-0.1) grid (40,5.1);
\draw[->] (-0.2,0) -- (41,0) node[right] {$x$};
\draw[->] (0,-0.05) -- (0,5.2) node[above] {$acc(x)$};
% just transformation
\draw[] plot[mark=*, mark options={yscale=0.4,xscale=2,color=red}] coordinates
{(0,1.0) (1,1.0) (2,1.0) (3,1.0) (4,1.0) (5,1.0)
(6,1.0) (7,1.0) (8,1.0) (9,1.0) (10,1.0) (11,1.0) (12,1.0) (13,1.0) (14,1.0) (15,1.0) (16,1.0) (17,1.0) (18,1.0) (19,1.0)
(20,1.0) (21,1.0) (22,1.0) (23,1.0) (24,1.0) (25,1.0) (26,1.0) (27,1.0) (28,1.0) (29,1.0) (30,1.0)
(31,1.0) (32,1.0) (33,1.0) (34,1.0) (35,1.0) (36,1.0) (37,1.0) (38,1.0) (39,1.0) (40,1.0)}
node[right,xshift=0.5cm,red]{\texttt{transform}};

% deforestation
\draw[] plot[mark=*, mark options={yscale=0.4,xscale=2, color=blue}] coordinates
{(0,3.0) (1,1.4) (2,1.3636363636363635) (3,1.2857142857142858) (4,1.303030303030303)
(5,1.2857142857142858) (6,1.2985074626865671) (7,1.2921348314606742) (8,1.3008849557522124)
(9,1.297872340425532) (10,1.304093567251462) (11,1.302439024390244) (12,1.3070539419087137)
(13,1.3060498220640568) (14,1.3095975232198143) (15,1.3089430894308942) (16,1.3117505995203838)
(17,1.3113006396588487) (18,1.3135755258126196) (19,1.3132530120481927) (20,1.3151326053042123)
(21,1.3148936170212766) (22,1.3164721141374838) (23,1.3162901307966706) (24,1.3176341730558598)
(25,1.3174924165824065) (26,1.3186504217432053) (27,1.318537859007833) (28,1.3195458231954582)
(29,1.3194549583648751) (30,1.3203401842664777) (31,1.320265780730897) (32,1.3210493441599)
(33,1.3209876543209877) (34,1.3216860787576261) (35,1.321634363541121) (36,1.3222607833415965)
(37,1.3222170032879286) (38,1.3227819884083816) (39,1.3227445997458704) (40,1.3232567513099556)}
node[right,xshift=0.5cm,yshift=0.5cm,blue]{\texttt{deforest}};

% supercompilation
\draw[] plot[mark=*, mark options={yscale=0.4,xscale=2, color=orange}] coordinates
{(0,3.0) (1,3.5) (2,5.0) (3,2.25) (4,1.7916666666666667) (5,1.6153846153846154)
(6,1.5263157894736843) (7,1.4743589743589745) (8,1.3363636363636364) (9,1.2534246575342465)
(10,1.238888888888889) (11,1.224770642201835) (12,1.2115384615384615) (13,1.1993464052287581)
(14,1.1882022471910112) (15,1.1780487804878048) (16,1.1688034188034189) (17,1.1603773584905661)
(18,1.1526845637583893) (19,1.1456456456456456) (20,1.1391891891891892) (21,1.1332518337408313)
(22,1.1277777777777778) (23,1.1227180527383367) (24,1.1180297397769516) (25,1.1136752136752137)
(26,1.109621451104101) (27,1.1058394160583942) (28,1.1023035230352303) (29,1.0989911727616646)
(30,1.0958823529411765) (31,1.092959295929593) (32,1.0902061855670102) (33,1.0876089060987415)
(34,1.0851548269581057) (35,1.082832618025751) (36,1.0806320907617504) (37,1.078544061302682)
(38,1.0765602322206096) (39,1.0746730901582933) (40,1.072875816993464)}
node[right,xshift=0.5cm,yshift=0.4cm,orange]{\texttt{supercompile}};

\foreach \x in {0,5,10,...,40} \draw (\x cm,1pt) -- (\x cm,-1pt) node[anchor=north] {$\x$};
\foreach \x in {0.5,1,...,5} \draw (1pt,\x cm) -- (-1pt,\x cm) node[anchor=east] {$\x$};
\end{tikzpicture}


These results demonstrate, that the speed-up from supercompilation is most impressive
for small $x$;
for such small numbers we do not reach the cases of generalization present in the graph.
When $х$ grows, however, the drawbacks of performing generalization start to show,
and the deforested task turns out faster.

%\begin{exercise}
%То, что суперкомпиляция может выдавать программы, работающие медленнее, чем дефорестированные,
%даже при ``идеальной'' модели стоимости вычисления (у нас скорость вычислений измерялась в шагах
%интерпретатора), было для меня поначалу большим сюрпризом. Задание объяснить, как такое возможно,
%достается в качестве награды самому любопытному и дотошному читателю, добравшемуся до конца этого
%опуса.
%\end{exercise}
