\documentclass{tmr}

\title{Editorial}

\author{Edward Z. Yang\email{ezyang@cs.stanford.edu}}

\begin{document}

This issue's Monad Reader clocks a record page count, with over a hundred
pages of Haskell goodness from nine authors.  Given that there's so much;
I thought this issue's editorial would be best devoted to giving teasers
for each of the articles herewithin.

\begin{itemize}
    \item Ever wanted to solve a programming interview question by writing
        a domain-specific language on the fly?  In \textbf{FizzBuzz in
        Haskell}, Maciej takes a whimsical look at the classic interview
        question, dancing through syntax, semantics and algebraic
        transformations to arrive at a solution that would be called
        elegant by most people.

    \item Ilya and Dimitur write in with a somewhat apologetic but
        very interesting tutorial about \textbf{Supercompilation}, a
        whole-program transformation technique.  It is a nice introduction
        to the subject, and comes with running code to boot.

    \item In a break from the usual uses of Haskell, Henrik, Daniel and Mikael
        describe \textbf{Haskell sound specification DSL} used in service of a live
        action role-playing game situated in the Battlestar Galactica universe.

    \item Alberto talks about his web framework, \textbf{MFlow}, which provides
        a continuation-like method of structuring web applications, without actually
        opening the can of worms that is explicitly serializing continuations.

    \item And last but not least, Neil offers a gentle introduction to
        \textbf{Practical Type System Benefits}, giving a taster of some of
        the more intermediate features of Haskell, including generic
        programming, parallelism, newtypes, quasiquoting and lightweight
        contracts.
\end{itemize}

Take a look!

\end{document}
