\section{Experiences}
\label{sec:experiences}

The experiences gathered in this project can be organized into the various
stages of the project.

\subsection{Rapid development and package designs}
\label{sec:rapid-devel-pack}

It turns out that the \texttt{amqp} package expects lazy bytestrings
and the \texttt{hedis} package expects strict bytestrings. The
\texttt{bytestring} package installed had no automatic way to convert
between these -- and it took a while in the project to figure out how
to make all systems interoperate.

Several features of the resulting sound scape daemon were
direct consequences of package and platform choices. The automatically
generated serialization and parsing capabilities, both in the
\texttt{Show}/\texttt{Read} dyad and in the \texttt{Generic}
generation of JSON and bytestring serializers all mean that parsing
and serialization worked automatically and immediately. Using the
callback structure of the \texttt{amqp} library also meant that a
reactive server loop was provided essentially for free.

All these features contributed to a rapid development with large
amounts of functionality arriving early in the process.

\subsection{Interfacing to other component platforms}
\label{sec:interf-other-comp}

Since the entire project was communicating by \texttt{AMQP} with JSON-encoded
packages already, the use of these standards made interoperability
easy. One of the most difficult parts of this side of the development
process was in inspecting emitted JSON packages to figure out the
particular encoding that the \texttt{Generic}-generated JSON parser
worked with, so that appropriate and parseable structures could be
emitted from the Ruby control interfaces.

Once the reactive interfaces through the \texttt{amqp} module were in
place, it quickly became clear to us that the easiest way to provide user
interfaces to the daemon would be through a Ruby web application
front end that emitted packages crafted to trigger specific rules in
our trigger system. 

\subsection{Interfacing to creative design teams}
\label{sec:interf-creat-design}

One original expectation that did not work out as expected was to
produce a language for other users to work with; as often is the case
with large scale projects relying on voluntary work and expansive
creative vision, work on the project progressed beyond the launch of
the first performance. The sound design teams worked on sound design
far beyond the stage where they would have needed to familiarize
themselves with syntax and functionality of the sound specification
system to use it themselves.

Instead, as a compromise, we requested all relevant information to
encode the designed soundscapes; and iterated through requests for
more information and educated guesses until soundscapes were defined
and could be tested. Here, the choice of embedding the language within
Haskell paid off; the sound specification programmer was able to use
many Haskell-specific idioms and structures to produce long sequences
of parametrized and repetitive sound scape definitions. These were in
the end stored as commands stored in \texttt{triggers}, listening to specific triggering packages,
often creating new triggers for later reactions. This enabled
constructions where a player action (pressing the Load Torpedos
button) would start some sounds (a loading torpedo noise), but then
wait for game master confirmations before playing further sounds (a
loading-finished noise and a robotic voice confirmation). The later
sounds would be temporarily installed trigger definitions waiting for
a confirmation package and including cleanup code that erased all the
temporary trigger conditions after sound execution, as described in
the examples above.

\subsection{Latency in performance}
\label{sec:latency-performance}

There were latency issues in the performance use. We had, in the end,
320 trigger expressions, each carrying up to 300 sound play events in
short sequences. At the first performance, the on-site crew noticed
latencies reaching up towards 8--10 seconds between triggering action
and sound execution. Before we were able to isolate the cause of these
issues, and while we were still comparing complexity properties of the
chosen containers -- Haskell lists and strict hashmaps depending on
whether inspection of all members would be a common operation or not
-- the performance issues decreased to within a few seconds. Given the
acoustic layout of the ship, this was deemed sufficient for further
performances. 

It should be noted that the high-latency results required some
dramaturgic edits; certain sound effects were not used because they
were dependent on low latency, and thus infeasible even with the
reduced performance issues. It was decided between the first and the
second performance to run with the system as built and not try for
specific further optimizations.

\subsection{Hardware Issues}
\label{sec:hardware-issues}
An issue arouse based on the hardware. Once the system was properly deployed and the project entered
the testing and verification stage, it was discovered that before playing sounds, the system emitted a few seconds of pops and noise regardless of the sound played. After researching the issue, this was proven to be a known issue with the RP platform, stemming from the way power supply is routed to the sound chips on the PCB\@. Known workarounds existed which were applied. \EZY{cite?} These resolved enough of these issues for the solution to be acceptable.

\subsection{Experiences on-site}
\label{sec:experiences-on-site}

The system was installed in the war ship two days before start of the game. Installation was done in three steps; preparing clients, server-side, and on-site configuration. Since circumstances at the location were more spartan than other, off-site locations available to the team -- most preparation was done in-office and then transported to the actual game location. 

Client preparation consisted of pre-loading SD cards for the Raspberry Pis with client daemons, the pre-configured sound library and connection bridges. A disk image was prepared and then duplicated onto the memory cards. This saved time at installation time, but proved to be a bad choice on-site. Certain updates could be performed over the network -- but due to issues with the disk image duplication the memory cards had to be reprogrammed. This could only be done with the physical cards in hand and since the Raspberry Pis were distributed over a large area and, in many cases, in locations hard or cumbersome to reach -- this proved to be a larger job than anticipated. Apart from this, pre-loading the memory cards worked very well. 

Server-side installation and configuration was done via \texttt{cabal} on the main game server. Haskell code was kept up-to-date via a Git repo and regularly synchronized from an online repository. Installation on server went as expected and was very easy to get set up. 

On-site configuration was designed to be done in three steps; physical
installation, soundscape configuration and verification. Physical
installation was cumbersome but not beyond expectations. Soundscape
configuration worked almost as expected. The \texttt{Commit} command built into the system proved invaluable since it allowed for one part of team working remotely to write an import-definition of the soundscape then send it to the on-site team, who imported it into the server and then made local backups and exports.

\section{Conclusion}
\label{sec:conclusion}

Haskell as a platform, coupled with programming techniques connected
to a functional reactive programming style, datatype construction
for domain modeling and a package collection that neatly covered all
required technological dependencies all combined to a successful
project where the sound specification system produced significant
value to the game experience and to the immersion into the game world
for the participants. Features of Haskell as a platform and of the
packages used produced a more capable deliverable than the original
specification had expected.

We are not releasing the source code for this system, however,
we encourage anyone interested in further details or derivative
projects to get in touch with any of the authors.


%%% Local Variables: 
%%% mode: latex
%%% TeX-engine: default
%%% TeX-master: "tmr"
%%% End: 
