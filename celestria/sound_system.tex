
\section{Game control system design}
\label{sec:game-control-system}

The technology team for \emph{The Monitor Celestra} settled at a
relatively early stage on an overall design for the game support
technology; the choices made included: \EZY{Work harder to have parallel construction in this bulleted list. It might be easier if you use complete sentences.}
\begin{itemize}
\item Wired ethernet communication for everything, split into three
  different parallel networks: sound, game system, and player hackable. Due to the nature of the 
  play area, parallel network backbones were installed for maximum durability and error management.
\item Communication between game components over \texttt{AMQP} from a RabbitMQ running on a local
game server on the main network. Game logic, world engine and game rules were written in Ruby.
\item Game consoles and GUIs written in HTML5 Canvas rendered on Chromium running in fullscreen.
\item All game consoles must be able to used interchangeably with custom hardware components being plug-and-play.
\item No authentication required within the game network. All published messages can be trusted on a system level. 
\item A single game component responsible for game world, time progression and synchronization; in this case, a rack server running latest stable Ubuntu LTS\@.
\item Connection to external network and Internet through a single point-of-entry, the main game server.
\end{itemize}

The choices were made based on experiences from earlier projects staged
in similar contexts; not-for-profit with volunteers, widest match against available skillsets, relatively uncontrolled environments
and subjected to the wear-and-tear of playing adults. \EZY{This sentence is really awkward}


\section{Sound system design}
\label{sec:sound-system-design}

The original specifications for the Celestra sound system were as follows:
\begin{itemize}
\item Each section/room of the ship should have an independent sound producer, so that each room could play different sounds simultaneously. Each sound producer should be able to mix several sounds.
\item It should be possible to trigger sound effects at specific
  locations, both automatically via the game control system and
  manually by game masters. For example, a torpedo sound in the
  torpedo room is played when a torpedo is launched (triggered by a player), and
  a warning sound is played in every room when the hull is compromised.
\item To maintain immersion, the sound effects must have low latency.
\item It should be possible to trigger sounds at certain future times. For instance, it should be possible to specify that a torpedo hatch closing sound is played exactly one second after the torpedo launch sound.
\end{itemize}

The rest of the game control system was designed to run on a
relatively small number of servers in a control room. One
option for the sound system would be to have a sound server with one
sound card for each section, connected to pairs of speakers located
in each section. However, it was decided that a better solution would
be to have actual computers at each section, with their own sound
cards, connected to a central computer via wired Ethernet, which would
already existed in most rooms. This setup would be easier to build in
terms of hardware and would be easier to program for. The computers
located next to the speakers in each room then had to be inconspicuous
and inexpensive, since quite a large number would be needed. This led us to go
for the Raspberry Pi (RP) setup.

The choice of RP added constraints. RP has limited RAM and CPU power;
hence, we decided to avoid adding dependencies to the game back-ends on
every RP\@. While most of the game control systems were written in Ruby
and communicated over \texttt{AMQP}, we decided to implement a
special purpose sound daemon in \Cpp, which would run on each RP and
communicate with the central sound server directly via sockets. This
would enable us to achieve lower latency and not drain resources from
the RP\@.

\subsection{Sound daemon}

All required sounds were stored as sound files on every RP,
effectively creating a global sound list $\mathcal{S}$. The sound
daemon should then accept a few simple commands on the socket:
\begin{itemize}
\item Play sound $k$ from $\mathcal{S}$ at absolute time $t$, as sound with ID $n$.
\item Stop sound with ID $n$, at absolute time $t$.
\item Change volume to $v$ on sound with ID $n$, at absolute time $t$.
\end{itemize}

The idea here was that by combining these commands, it would be
possible to realise all the required sound system features. Each RP was time
synced to the rest of the network using NTP~\cite{ntp}, hence the
sound commands could have absolute timestamps with high accuracy. To
achieve the possibility of playing multiple sounds at a single RP, the
sound daemon communicated with PulseAudio~\cite{pulseaudio}, creating
one stream per sound being played. To achieve low latency and to
simplify the implementation, the daemon read in $\mathcal{S}$ to RAM
upon start-up, rather than reading a file upon each play command. This
added the constraint that the total uncompressed size of the sounds
had to fit into RAM of the RP\@.

\subsection{Ambient and story sounds}

It turned out that apart from the sound effect type of sounds that
the sound daemon was created for, the game required two other types of sounds.
\begin{itemize}
\item Ambient sounds, which should be looped in the background in a section. For instance, in the reactor room there should be a constant sound from the reactor.
\item Story sounds, which were played at specific events in the game. These were typically very long compared to ordinary sound effects and were triggered explicitly by the game masters.
\end{itemize}

These types of sounds did not quite fit into the sound daemon. The
ambient sounds required looping and needed to be cross faded into
themselves to avoid clicks; the story sounds were too large to fit
into RAM\@. Moreover, the low latency requirement for these
sounds were much less strict.

We therefore decided to use MPD~\cite{mpd} for these sounds. MPD reads
files from disk, hence the sound size would not matter. It also has built-in
cross fade and loop support and can output to PulseAudio, so it was compatible with the sound daemon. Each RP was running an MPD daemon,
and the sound server communicated with it using a standard MPD client library.

\subsection{Sound daemon controller}

On the sound server, a small piece of controller software was running, which listened
to AMQP sound commands, transformed them into sound packets for the
daemon or to commands to MPD and sent them off to an RP\@. It was also responsible for transforming node numbers to IP addresses of RPs.

Like most of the game control software, this controller was written in
Ruby. The original idea was that this would be the entry point of the
sound system. However, it turned out that its simple interface, while
in principle sufficient for all sound purposes, was too low-level for
creating the sound scenes that the sound designers wanted to build for
the game: crossfades, seamless loops and other sound design features
would require a higher abstraction than the play/loop/stop/change
volume setup for this interface

Thus, we were led to introduce the intermediate layer in Haskell,
described in the following sections. From now on, we will be referring
to the sound server described here as the \textbf{low-level system},
and to the Haskell layer as the \textbf{high-level system}. The
Haskell layer works with two separate interface layers; a
\textbf{low-level interface} facing the low-level system, and a
\textbf{high-level interface} facing the game master and sound
designer teams.


%%% Local Variables: 
%%% mode: latex
%%% TeX-engine: default
%%% TeX-master: "tmr"
%%% End: 
