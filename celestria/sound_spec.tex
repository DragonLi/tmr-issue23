\section{Requirements on the sound scene DSL}
\label{sec:requ-sound-spec}

The work on building a dedicated domain specific language
(DSL) for describing sound scenes
started at a relatively late stage of the project, when most other
platform decisions had already been made. The need for an intermediate
layer emerged from the simultaneous need for low-level simple sound
execution protocols that could run on a Raspberry Pi platform and for
abstract sound descriptions with a low technical usage barrier to
enable the game fiction design and sound design crews to interact with
the system.

This setting produced a number of requirements that we tried to adapt
to during the project:
\begin{description}
\item[Simple output] Output from the DSL to the lower level systems
  needed to have a simple structure, preferably consisting of discrete
  orders to \textsc{Play}, \textsc{Loop}, \textsc{Stop} or
  \textsc{Adjust} sounds at sound nodes (both addressed by integers).
\item[Accessible input] The DSL needed an easy to read and easy to
  write format that the artistic side of the project could use to
  specify sound scenes.
\item[Fast development] The DSL needed to be developed during 4 months
  of volunteered free time, during a time span that included several
  major conference deadlines. This limited the amount of programmer
  manpower available.
\item[System compatibility] The DSL needed to react to \texttt{AMQP} messages,
  store state in \texttt{Redis}, and format its output in a way easy to parse
  by the receiving low-level and high-level interfaces.
\end{description}

Due to earlier experiences with Haskell, we chose GHC and the Haskell
Platform as a host system for the DSL development. An early decision was
to use the automatic parser generation in the standardized Haskell
\texttt{Read} and \texttt{Show} classes; later on, we used the
\texttt{Generics} extension to automatically generate JSON parsers and
formatters in order to plug into the existing \texttt{AMQP} message
passing architecture. This reduced large amounts of DSL design to
designing appropriate Haskell types: writing native Haskell code with
custom-built data types that represented the domain allowed for large
freedom in defining our own semantics while retaining the full power
of Haskell packages and programming styles.

%%% Local Variables: 
%%% mode: latex
%%% TeX-engine: default
%%% TeX-master: "tmr"
%%% End: 
