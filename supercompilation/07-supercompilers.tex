\section{Existing supercompilers}
\label{sec:supercompilers}

% Для всех суперкомпиляторов очень не хватает описания предела их возможностей:
% что во что они могут превратить, а что - нет?
% Какие из них наиболее <<практичны>>? Применялись ли какие-либо из них
% для практических задач?

We list here some existing supercompilers.
The list is not exhaustive; it contains implementations with interesting features,
publicly accessible,
and which are either actively developed, or at least have been until recently,
so there is some greater chance to make them work.
Note that all these implementations are considered experimental.

\begin{longitem}

\item \emph{SCP4 \cite{Scp4Url}.} SCP4 
is the latest incarnation of the first ever supercompiler, which was
developed for the Refal language. 
It supercompiles programs in a recent version of Refal, Refal-5.
SCP4 utilizes some of the features of Refal, which make it particularly
suited to supercompilation, like associativity of sequence concatenation.
It also features a number of extensions of the basic super-compilation method:
recognition of constant functions,
recognition of concatenation monomials,
collection and analysis of output formats.
% выглядит как набор специфических трюков (как rule-based оптимизация в обычных компиляторах).
% это фундаментальные преобразования или действительно трюки?
% нельзя ли в принципе сделать то же без трюков? казалось бы, именно в этом одна из сторон
% привлекательности суперкомпиляции по сравнению с традиционным подходом.
SCP4 is described in detail in \cite{Nemytykh2008PhD} and in the monograph \cite{Nemytykh2007SCP4}.
SCP4 can extend the program domain in the following sense:
if the input program loops or stops with an error on certain inputs,
it is sometimes possible for the residual to successfully terminate on these inputs.

\item \emph{A (nameless) supercompiler for the  TSG language \cite{TsgUrl}.} TSG 
is a greatly simplified version of List, which is ``flat'' (no nested function calls are allowed).
Besides the experimental supercompiler for TSG described in \cite{Abramov2006meta2},
there exist implementations for a number of other methods, which have arisen 
in the context of supercompilation, and which, unfortunately, get less attention
than they deserve.
These methods include neighborhood analysis, neighborhood testing, inverse programming,
nonstandard semantics \cite{Abramov1995meta}.
% либо сказать, что это такое, либо дать ссылки

\item \emph{Jscp\cite{Klimov2008Jscp} \cite{JscpUrl}.} A supercompiler for Java.
The first supercompiler for an object-oriented, real-world language.
One of the main conclusions of this experiment: supercompilation of
non-functional languages is much more complicated.
Jscp, unlike most other supercompilers in this list, is closed-source.

% http://code.haskell.org/timber/src/Scp.hs
\item \emph{A supercompiler for Timber \cite{TimberUrl}.} Timber 
is a pure object-oriented language with call-by-value semantics, mostly inspired by Haskell.
Its supercompiler treats the purely functional subset of the language.
The main goal of this project is to achieve similar optimizations 
when supercompiling a call-by-value language -- compared to a call-by-name language --
while fully preserving the semantics of the original program.
Good descriptions exist in \cite{Jonsson2008Supercompilation,Jonsson2011Phd}.

\item \emph{Supero \cite{SuperoUrl}.} A supercompiler for a subset of Haskell.
Supero is the first supercompiler to treat a call-by-need language,
which was actually the main goal of the 
project \cite{Mitchell2008taa,Mitchell2010Rethinking}.

\item \emph{SPSC \cite{SpscUrl}.} SPSC is another toy supercompiler for SLL.
The main goal was to implement positive supercompilation,
as described (but not implemented) by 
S{\o}rensen \cite{Sorensen1996Positive,Sorensen1998Introduction}. 
SPSC is described in \cite{Klyuchnikov2009SPSC}.

\item \emph{HOSC \cite{HoscUrl}.} HOSC is
a supercompiler for a (call-by-name) Haskell subset.
Unlike many other projects, where the main goal is program optimization,
here the main interest lies in program analysis by supercompilation.
Besides standard supercompilation, HOSC can also perform
``two-level'' supercompilation, based on discovering and applying 
``improvement'' lemmas \cite{Klyuchnikov2010Phd,Klyuchnikov2010Fast}.

\item \emph{Optimusprime \cite{OptimusprimeUrl}.} Optimizes
functional programs, which are then run on a specialized FPGA-build
processor (Reduceron) \cite{Reich2010Reduceron}.

\item \emph{CHSC \cite{ChscUrl}.} Another
supercompiler for a Haskell subset \cite{Bolingbroke2010Eval}.
The main goal of this project was to include supercompilation
as an optimization pass inside GHC.
An important technical difference is that CHSC does not perform
lambda-lifting as a preprocessing step, unlike most other
supercompilers for higher-order languages.

\item \emph{Distillation \cite{DistillUrl}.} Most
supercompilers use configurations based on expressions with free variables.
Distillation uses configurations, each of which in turn is 
similar to a configuration graph. 
In other words, the nodes of distillation's configuration graphs contain nested graphs;
and folding and generalization must be defined over graphs. 
Such complications are the price for obtaining more powerful
optimizations, compared to standard 
supercompilation \cite{Hamilton2007Distillation,Hamilton2010Graph}.

\item \emph{MRSC \cite{MrscUrl}} A toolkit 
for building ``multi-result'' supercompilers.
Multi-result supercompilation \cite{klyuchnikov2011mrsc} is another generalization of the standard
supercompilation technique, where the supercompiler is permitted to
return multiple residual programs (all correct, of course).
While this may seem useless at first glance, it turns our that 
this extension opens the way for a much more flexible and modular
description of supercompilation, and in the same time permits
some optimizations, which are out of reach for the standard methods.

\item \emph{A Coq framework for building formally verified supercompilers \cite{CoqGenScpUrl}} 
A framework -- in Coq -- for building modular supercompilers by just defining 
the basic ingredients (configurations, driving, folding, etc.),
and plugging them into a prefabricated generic supercompiler \cite{TFP2012:CoqGenScp}.
The organization of the supercompiler is very similar to the one we have just described,
and there is a complete example supercompiler for a language very close to SLL.
The main advantage of the framework is that it facilitates formally establishing
the correctness of each new supercompiler, by just having to prove some simple properties
concerning the basic building blocks.

\item \emph{A supercompiler for Erlang \cite{ErlangScpUrl}} A recent
first step towards a practical supercompiler for another popular language -- Erlang.

\item \emph{TT-Lite \cite{TTliteUrl}} A supercompiler
for a version of Martin-L\"{o}f's Type Theory (MLTT).
TT-Lite is the first supercompiler for a terminating (non-Turing-complete) language
with dependent types \cite{Klyuchnikov2013TTLite}. 
Another interesting feature is the generation of ``certificates'',
containing formal, automatically verifiable evidence 
that the residual program is equivalent to the input one, in each specific instance. 
The certificates themselves are encoded as MLTT terms.

\end{longitem}
