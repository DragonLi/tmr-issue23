\section{Article Goals}
% Сюда бы добавить 1 предложение вроде "Суперкомпиляция - это ...",
% ибо 1) для человека, впервые слышащего о ней, немного огорошивающий эффект от текста,
% ИК: добавил в аннотацию

% В последние три-четыре года заметен рост интереса к идеям суперкомпиляции в научном (больше
% западном, чем в российском) сообществе, занимающемся функциональным программированием.
%

%\EZY{Awkward sentence construction. Can you make it shorter/more punchy?}
Real-world functional programmers, who have heard something about supercompilation,
express very varying attitudes.
The most common reaction is one of mistrust or even full rejection, 
mostly because of the fact that there is no industrial-strength supercompiler yet. 
All existing supercompilers have an experimental status
and, in most cases, are only successfully used by their own authors.

Supercompilation is, in a certain sense, a very general method, but, in practice, only
specialized tools -- re-using just some of the ideas of supercompilation -- work well
enough to be useful. 
This clashes with the expectation that a general tool should come out of a general idea.
Another possible reason for the mistrust is the fact that supercompilation is still lacking a ``killer application''.
% \footnote{Попробую привести здесь такую
% аналогию. Есть общая идея распараллеливания существующих программ. И есть хорошие узко направленные
% инструменты, распараллеливающие некоторые программы очень даже хорошо. Универсального
% распараллеливателя пока что тоже нет.}.

One common misconception persists: that supercompilation is only a tool for program optimization.
Supercompilation is a \emph{program transformation technique}.
Program transformations can have different goals.
One such goal can be optimization, another -- program analysis.
Supercompilation is equally applicable to both these goals.
Most works dealing with supercompilation consider only the optimization aspect,
but we should not be misled that optimization is the only useful application of supercompilation.

We must distinguish the notions of \emph{supercompilation} and \emph{a supercompiler}.
Many articles describe various technical difficulties arising from the implementation
of a specific supercompiler and ways to overcome these difficulties, while mostly
leaving aside the underlying key ideas of supercompilation.
In most cases, the parts of a given supercompiler interact in intricate ways, which
may give the impression of a complicated monolithic 
construction.
Or, as Simon Peyton-Jones put it: 
``\ldots To build a supercompiler, if you look at how it works, there are a number of things all wound together in one rather complicated ball of mud. 
I found it extremely difficult when I really wanted to understand it. 
I found it very difficult to understand the papers. \ldots''
(in an interview for ``Practice of Functional Programming'', \cite{Jones2010Interview}).
%\EZY{Can we get a proper cite here?}

The main goal of this article is to give a clear and concise illustration of the
aforementioned Turchin's quote, describing the essence of supercompilation
by using a minimalistic example supercompiler.
The accent is on delineating and explaining the basic building blocks
and showing how these blocks can be implemented and made to work together
in a clear and simple fashion.

\textbf{Organization of the article:}
A important companion to this article is the full source code of the toy supercompiler
SC~Mini, with detailed comments \cite{SCMiniUrl}
(250 lines of main code + 200 lines of auxiliary utilities in Haskell).
The text of the article itself can be seen as an introduction to studying
these supercompiler sources.
The article introduces and explains the main supercompiler terminology and ingredients,
and we illustrate SC~Mini's behavior on suitable examples.
These examples are chosen to be neither too complicated, nor oversimplified.
The reader is thus invited to carefully study all examples.

% Статья описывает идеи и методы
% суперкомпиляции и демонстрирует некоторые ее возможности на простых работающих примерах. Рассмотренные методы
% реализованы автором в учебном суперкомпиляторе SC Mini, написанном специально для данной статьи. Сам
% суперкомпилятор SC Mini и его подробное описание
% даны в дополнительных материалах, которые доступны на сайте журнала. Приложение может представлять
% интерес для программиста на Хаскеле, желающего ближе познакомиться с техническими аспектами
% устройства суперкомпилятора.

\textbf{Things left out:} 
Do not expect to find a detailed comparison of supercompilation to other methods
of program optimization/transformation -- it would require a survey article
of a much larger scope. 
A comprehensive bibliography of supercompilation literature is similarly out of scope,
although we do give many useful references where relevant.

